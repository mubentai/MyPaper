% !Mode:: "TeX:UTF-8"

\chapter{总结与展望}
本论文主要在纯电动物流汽车采用的磷酸铁锂电池方面做了相关的优化研究,为锂电池荷电状态估计领域做出了一定的贡献。但同时本论文针对锂电池SOC 实时动态估计方面的研究也有不全面的地方,导致如果想将本理论直接用于纯电动物流汽车或者相似车型的SOC估计还需要进一步的研究和更高规格的硬件支撑。
\section{总结}
随着国家大力推动新能源汽车在城市交通、物流等领域的发展,城市纯电动物流汽车项目的研究从一开始就备受企业和高校研究者的关注。而一辆纯电动物流汽车的制造牵涉面广泛,不仅涉及到机械制造工艺,而且与锂电池材料工艺、发动机制造等方面相关。在复杂的纯电动物流汽车设计中,锂电池组作为动力来源具有不可替代的作用,也因此成为纯电动汽车研究的重点。由于锂电池组材料研究的瓶颈性,所以在既有的锂电池的基础上研究高效的电量损耗以及电量控制就显得尤为重要,具有实际的经济价值并且能够提高纯电动汽车的安全性。

首先本论文立足于纯电动物流汽车的电池进行研究,选取了电池管理中最重要的参数——电池的荷电状态作为研究对象。论文一开始就深入研究了磷酸铁锂电池的工作原理和基本特性,通过锂电池的正负极反应对电池的电化学模型进行了表述,考虑了锂电池发生化学反应时锂离子的迁移扩散过程以及锂电池的浓差极化和电化学极化现象。在对电池容量特性、内阻特性、开路电压特性和温度特性的总结下引出表征锂电池的等效电路模型。综合等效电路模型的复杂度以及阶次的选取,选定二阶RC的等效电路模型作为本论文研究的锂电池模型支撑。然后通过分析锂电池在实际反应中表现出来的分数阶特性,将分数阶微分理论引入到电池等效电路模型的求解中,在计算等效电路的电压电流分数阶微分关系时,通过Oustaloup离散递推滤波器对本文的分数阶模型进行整数阶的近似求解。然后通过参数辨识实验对本文提出的分数阶等效电路模型的特征参数进行估计表示,为锂电池的荷电状态估计提供基础支撑。本论文在锂电池分数阶动态等效电路模型方面的研究对于锂电池模型的表征具有实际的意义,起到了良好的推动作用,希望能够促进分数阶微积分理论在锂电池领域的研究。

其次,本论文对磷酸铁锂电池的荷电状态估计算法进行了研究。在锂电池分数阶等效电路模型的基础上,分析了基于卡尔曼滤波的荷电状态估计方法。从卡尔曼滤波器的基本原理入手,从线性卡尔曼滤波器的思想切入,对比分析扩展卡尔曼滤波和无迹卡尔曼滤波对于非线性系统处理的不同特点。对扩展卡尔曼滤波采用泰勒展开式进行线性化处理时引入的误差进行了分析,并且在此基础上提出了一种基于分数阶模型的无迹卡尔曼滤波算法。该算法以提高荷电状态估计的精度为研究目标,避免了扩展卡尔曼滤波由于线性化而引入的误差。该算法融合了分数阶微积分的动态特性表征能力以及UT变换的近似能力,构建了一种高精度的锂电池荷电状态估计算法。并且还对扩展卡尔曼滤波和无迹卡尔曼滤波算法进行了实际的仿真分析,以实验结果来表征估计算法的优越性和精确度。

最后为了验证本文提出的荷电状态估计算法的性能,以成都王牌集团生产的纯电动厢式运输车(CDW5070XXYH1PEV)为原型,在ADVIDOR车辆仿真平台下搭建了纯电动物流汽车的整车模型,并且在典型的道路工况下对车辆电池能量的消耗进行仿真分析,同时与磷酸铁锂电池的荷电状态估计结果进行对比,表明基于无迹卡尔曼滤波的荷电状态估计算法具有较为准确的性能和较高的精度。
\section{展望}
无论是企业还是科研单位对于纯电动汽车锂电池荷电状态估计方法的研究已经取得了阶段性的进步,取得了很多丰硕的成果,并且也有部分研究理论投入到实际的生产当中。但是总的来说,从理论到实际应用之间还有很大的鸿沟需要逾越,这是实际的工程条件所决定的。本文对磷酸铁锂电池的等效电路模型以及锂电池荷电状态的估计算法进行了研究,但是对于算法精确度的提高和算法实际工程运用还需要进一步的研究。

首先在于电池模型相关的方面,本论文采用的是二阶RC的等效电路模型。但是毕竟二阶模型的动态表征能力是有限的,所以在计算能力允许的情况下可以采用三阶或者更高阶的等效电路模型来表征锂电池的状态,以提高模型的准确度。本文采用的是分数阶微积分的Caputo定义,在表征锂电池模型时会存在相应的片面性,因此根据分数阶微积分的G-L定义或者R-L定义来进行锂电池等效电路模型的描述也是值得研究的。并且对于分数阶微分电路方程的求解,本文采取的是整数阶传输方程近似的方法,对于分数阶微分等效电路方程的直接求解或者间接求解也是值得进一步研究的方向。

其次在锂电池荷电状态估计算法方面,本文提出的基于分数阶的无迹卡尔曼滤波算法还存在不足的地方可以进行改进,在UT变换中粒子的选取也存在不同的方法可以尝试,在计算量允许的情况下可以引入粒子滤波等方法来提高估计算法的精度。

最后本论文在整车方针模型的搭建上主要是对车身系统和动力系统进行了修改,对于特定的车型和要求可以进行各个车辆模块更为细致的调试。并且本文选用了一种城市工况来对整车的能耗和电量消耗进行仿真实验,在实际需求下可以进行多工况的对比实验或者对需要的工况进行定制的修改来达到研究的目的。所以无论是对于锂电池模型的研究还是荷电状态估计算法的改进都存在进一步改进的空间与价值。