% !Mode:: "TeX:UTF-8"

\chapter{基于ADVISOR的SOC估计算法仿真实验}
在上文中对单体的磷酸铁锂电池在MATLAB下进行了基于无迹卡尔曼滤波算法的SOC估计仿真实验,但是对于实际的纯电动物流汽车的电池组是由180串44并个单体锂电池组合而成的,所以需要在道路工况下对纯电动物流汽车的SOC估计算法进行检验。在本章中,引入ADVISOR的继承仿真软件对基于无迹卡尔曼滤波的SOC估计算法进行模拟实验。
\section{ADVISOR仿真软件基本介绍}
先进的车辆仿真工具ADVISOR是一款基于MATLAB和Simulink软件环境的专业汽车仿真工具,它于1994年由美国国家可再生能源实验室首次开发使用\citeup{wipke1999advisor},可以对汽车的各个模块组成部分进行仿真分析,目前被广泛用于汽车仿真技术的研究。ADVISOR于2003年被AVL公司收购,并且AVL公司也停止了对该软件的更新,目前最新的版本是ADVISOR 2004,本文也在ADVIDOR 2004的集成仿真环境下进行整车模型的搭建和SOC估计算法的仿真。
\pic[htbp]{ADVISOR2004仿真平台}{}{4-ADVISOR}

采用模块化的设计理念是ADVISOR仿真平台的显著特点,ADVISOR主要包括了发动机模块、变速器模块、车轮车轴模块、减速器模块和离合器模块等模块,模块之间通过接口传递数据的输入和输出,提高了模块的内聚性,降低了模块之间耦合度。ADVISOR各个模块元器件都采用形象的图标进行标示,在Simulink中建立整车的动态模型时只需要将需要的模块拖动到界面中然后互相连接起来。ADVISOR仿真程序的模型和源代码是开源的并且许多部件都具有十分成熟的模型值得借鉴,这样可以方便研究人员对模型进行修改以达到深度定制的目的。
\pic[htbp]{ADVISOR2004仿真平台}{}{4-whole}

汽车仿真系统主要分为后向仿真系统和前向仿真系统,ADVISOR属于后向仿真系统,它的特点是从道路的实际工况出发而不是从驾驶员的操作意图出发来建立汽车的整车模型。同时ADVISOR仿真软件为用户提供了GUI界面的操作,方便用户进行模型参数的修改与添加。ADVISOR不仅可以应用于传统汽车领域,而且在纯电动汽车和混合动力汽车领域等新兴汽车领域也得到了广泛的应用和长足的发展,可以实现在实际的道理工况下对汽车的传动性能、能量消耗和功率损耗等特性进行仿真研究。
\section{基于ADVISOR的纯电动物流汽车的整车模型搭建}
\begin{enumerate}
\item 纯电动物流汽车车辆原型

本论文以王牌汽车集团生产的纯电动厢式运输车(CDW5070XXYH1PEV)为原型来进行ADVISOR整车模型的搭建,表\ref{vehicle}是原型车辆的参数列表。

\longthreelinetable{vehicle}{纯电动物流汽车整车参数}{3}{lcr}
{
序号&项目&参数值\\
}{
1&	车辆名称&	纯电动厢式运输车\\
2&	车型号&	CDW5070XXYH1PEV\\
3&	车长(mm)&	5995\\
4&	整备质量(Kg)&	3900\\
5&	纯电续驶里程(km)&	200(空载)\\
6&	最高车速(km/h)&	80\\
7&	额定载重(Kg)&	2600\\
8&	动力电池总质量与整备质量比值($\% $)&	22.6\\
9&	发动机型号(电机)&	YPQ200L-8-HT\\
10&	发动机厂家(电机)&	 江西特种电机股份有限公司\\
11&	发动机类型(电机)&	交流电机\\
12&	发动机额定功率(kw)(电机)&	55\\
13&	发动机峰值扭矩(N.m)(电机)&	1000\\
14&	货箱尺寸(mm)(内尺寸)&	4150*1910*1850\\
15&	电量、GPS信息获取方式 &	75kWh、GPRS\\
16&	最大爬坡度&	20($\% $)\\
}
\item 搭建车身整体模块

在ADVISOR的Vehicle Input界面对车身模型的参数进行选择性输入,由于本文的研究对象是纯电动物流汽车,所以选择的是车身中的$VEH\_EV1$模块,并且根据纯电动物流汽车的特性对模块的参数进行修改配置,参数值如表\ref{vehiclewhole}。

\longthreelinetable{vehiclewhole}{整体模块参数表}{2}{lcr}
{
参数值&参数说明\\
}{
$veh_gravity=9.81$  &  	加速度常数($m/{s^2}$)\\
$veh_air_density=1.2$  &	空气密度常数($kg/{m^3}$)\\
$veh_glider_mass=655$ &	车身重量(kg)\\
$veh_CD=0.19$ &	空气的阻力常数\\
$veh_FA=69.5*2.54/100*50.5*2.54/100*0.9$ &	车辆的迎风面积(${m^2}$)\\
$veh_front_wt_frac=0.55$ &	前轴载荷分布比例系数\\
$veh_cg_height=0.4$ &	车辆重心的高度(m)\\
$veh_wheelbase=98.9*2.54/100$ &车轴之间的距离(m)\\
$veh_cargo_mass=136$ &	载重货物的重量(kg)\\
}
\item 汽车电动机模块的构建

在ADVISOR中电动机模块(Motor)下拉列表中选中适合于锂电池组驱动的$MC\_AC124\_EV1\_draft$电机,并且对电机模块的配置参数做适用于物流汽车的修改,如表\ref{Motor}。

\longthreelinetable{Motor}{发动机模块参数表}{2}{lcr}
{
参数值&参数说明\\
}{
$mc_max_crrnt=480 $   	&电机最大的转动电流(A)\\
$mc_min_volts=120 $	&电机最小工作电压(V)\\
$mc_inertia=0 $	&电机的转动惯量($kg * {m^2}$)\\
$mc_mass=91 $	&电机的质量(kg)\\
$mc_map_spd=[0 1000… 10000]*(2*pi/60)$ & 电机的转矩范围\\
$mc_map_trq$ &电机的转速范围\\
}
\item 修改电动汽车锂电池模块

根据本文所研究的磷酸铁锂电池组的电压和电流特性,在ADVISOR的动力电池的下拉列表中选择$ESS\_PB65\_FocusEV$型号的电池组。然后对电池组的参数进行相应的修改,如表\ref{Battery}。

\longthreelinetable{Battery}{锂电池模块}{2}{lcr}
{
参数值&参数说明\\
}{
$ess\_soc = \left[ {0:.2:1} \right]$ &   	电池SOC的向量\\
$ess\_tmp = \left[ {0{\rm{ }}22{\rm{ }}40} \right]$ &	电池温度向量\\
$ess\_min\_volts = 9.5$ &	电池的放电截止电压(V) \\
$ess\_module\_mass = 19.490$ &	电池的重量(kg)\\
$ess\_module\_num = 28$ &	锂电池的节数\\
$ess\_cap\_scale = 1$	&电池模块最大容量因子\\
$ess\_th\_calc = 1$ &	电池是否有热效应计算\\
$ess\_mod\_cp = 660$	&电池模块平均的热容量\\
$ess\_set\_tmp = 35$	&电池模块的恒温温度\\
$ess\_mod\_sarea = 0.4$ &	电池空气冷却总模面面积\\
$ess\_mod\_airflow = 0.07/12$ &	电池冷却空气的流量\\
$ess\_mod\_flow\_area = 0.004$ &	电池每个模块的冷缺空气流通面积\\
$ess\_mod\_case\_thk = 2/1000$ &	电池模块的外壳厚度\\
$ess\_mod\_case\_th\_cond = 0.20$ &电池外壳材料导热系数\\
}
\item 实验所采用的循环道路工况

在ADVISOR的仿真环境中可以建立车辆运行的实际道路工况,该工况包括了车辆在行驶过程中的起步、加速、制动、上坡等行为,基本上包含了车辆在实际的道路上行驶时遇到的全部情况,对于算法实际应用的验证具有很好的参考价值。
\end{enumerate}

在实际整车模型的搭建中除了以上提到的五点,还存在车轮模块、传动系模块等相关模块,各个模块参数的调试与修改也是一个复杂并且繁琐的过程。总体的电动物流车整车模型如图\ref{4-vehicle}所示。
\pic[htbp]{整车模型参数界面}{}{4-vehicle}
\section{纯电动物流整车在实际工况下的SOC估计仿真验证}
以上一节在ADVISOR下建立的纯电动物流汽车整车模型为基础,本节主要选用实际的道路工况($CYC\_UDDSHDV$)模拟物流汽车在道路上行驶的实际状态,定量的分析本文提出的锂电池荷电状态估计算法的估计效果和精度。

UDDSHDV工况全称Urban Dynamometer Driving Schedule for Heavy-Duty Vehicles,代表重型货车城市道路运行工况。该工况可以准确的描述重型汽车在城市道路行驶中各时间点的速度、加速度和爬坡力度等信息,用于本文城市物流汽车实际行驶情况的实验十分贴切。重型货车城市道路工况的参数数据如表\ref{road}所示。

\longthreelinetable{road}{重型货车道路工况统计参数表}{2}{lcr}
{
工况参数&参数值\\
}{
行驶时间&1060s\\
行驶里程&8.94km\\
最大速度&93.34km/h\\
平均速度&30.32km/h\\
最大加速&1.96$m/{s^2}$\\
最大减速&-2.07$m/{s^2}$\\
平均加速&0.48$m/{s^2}$\\
平均减速&-0.58$m/{s^2}$\\
停车时间&353s\\
}

该道路工况的实时速度曲线如图\ref{4-Road}和图\ref{4-Speed}所示。
\pic[htbp]{货车工况实时速率曲线}{}{4-Road}
\pic[htbp]{整段工况速率分布图}{}{4-Speed}

从工况速率图标可以看出在城市道路行驶时速率一般不超过90km/h,并且有接近40$\% $的时间都集中在拥堵的情况下,这与纯电动物流汽车的实际行驶的工况是相符的。

在实际工况下,结合第三章提出的无迹卡尔曼滤波算法,进行普通模型和分数阶模型的对比实验,得到电池SOC估计曲线如图\ref{4-Result}所示,以及估计的误差如图\ref{4-Error}所示。
\pic[htbp]{分数阶模型UKF实验对比图}{}{4-Result}
\pic[htbp]{估计误差对比曲线}{}{4-Error}

从实验结果可以看出,在普通模型下进行无迹卡尔曼滤波SOC估计的误差在5$\% $左右,而在基于分数阶模型下进行无迹卡尔曼滤波SOC估计的误差这可以降低到3$\% $。虽然两者算法的估计效果均会随着时间的增长而有所降低,但是可以明显的看出分数阶模型的估计效果变坏趋势更加的平坦,所以基于分数阶的无迹卡尔曼滤波SOC估计算法具有更高的精度,降低了估计误差。

\section{本章小节}
本章在对ADVISOR车辆集成仿真环境充分熟悉的基础之上,研究了纯电动物流汽车在该平台下的各个模块的搭建以及参数的匹配优化,并且在搭建好的物流汽车整车模型之上,选取了典型的城市道路循环工况,然后将第三章提出的扩展卡尔曼滤波算法和无迹卡尔曼滤波算法在该仿真系统上进行对比的实验验证。通过实验的结果可以看出,在实际复杂多变的行驶工况下,在分数阶电池模型基础上构建的无迹卡尔曼滤波SOC估计算法比在同一模型下构建的扩展卡尔曼滤波SOC估计算法具有更高的精确度。