% !Mode:: "TeX:UTF-8"

\chapter{绪论}
\section{选题背景及意义}
随着能源逐渐枯竭和自然环境不断恶化,节能减排成为了当前国际工业发展的整体思路,新能源汽车产业的发展已经涉及到国家经济战略的部署,新能源汽车是未来汽车发展前进的方向,推动新能源汽车的发展对于减少有害气体和粉尘的排放具有积极的作用,对于改善我国现有的能源消费结构以及带动新能源相关企业的发展具有积极的指导意义。纯电动汽车也得以在推广新能源的大环境下迎来快速的发展,纯电动汽车由于自身电池组容量的限制续航里程成为制约其发展的一个重要难关,一般的纯电动汽车的续航里程在200km左右\citeup{zhang2014},然后就需要在充电桩进行充电,而由于大多数的充电桩都设置在城市内,所以纯电动汽车在城市交通的推广会具有较小的阻力。同时国家政策也指出了相应的发展方向,根据国务院交通部2015年发布的关于新能源汽车在交通运输业发展的指导意见显示,国家积极的鼓励城市新能源汽车发展,同时也为新能源汽车在城市公共交通领域以及物流等行业设定了明确的目标。因此纯电动物流汽车的研究也得以乘上政策的东风快速发展。城市物流配送在运用纯电动技术方面拥有得天独厚的优势,首先与大部分家庭使用的小轿车相比,物流配送车行驶的路线通常是提前规划好的,因此行驶的距离会相对比较固定,从而可以达到事先回避里程约束和充电约束的问题。并且纯电动物流配送车补给是通过充电完成的,减少了在燃油费用上的消耗,有效的降低了物流企业运营的成本,增加企业的收益,真正的推动了绿色物流理念的发展。因此,纯电动物流车综合其成本低廉、环保清洁等优势,将具有良好的发展前景。

	动力电池是影响电动汽车推广的重要因素,电池的使用寿命长短、充电时长、能量密度大小、价格等一系列因素都成为了影响纯电动汽车发展推广的关键技术难点。市场对于解决这些关键问题的需求十分迫切,这些问题的解决对于提高电池的利用率和降低电动汽车的成本起着至关重要的作用。这些参数指标都与锂电池荷电状态估计直接相关,所以在纯电动汽车在各个方面的研究中,电动车电池SOC(State of Charge)\citeup{piller2001methods}的精确估计是最基本、最重要的方面,为纯电动汽车的安全行驶、延长动力电池的使用寿命提供保证\citeup{zhoufeikun2013}。因此对于纯电动物流汽车SOC估计的研究成为重要的课题,具有实际的意义。
对于纯电动汽车而言,依赖动力电池组驱动的电机是其唯一的直接动力来源。纯电动汽车前进的驱动力来自由锂电池组提供能量的电动机,汽车的电动机具有额定的工作功率,同时在短时间内电动机也可以超负荷以峰值功率进行运转。电动汽车低速行驶时,电动机的输出转矩可以达到最大,功率上升,电池组电流增大,满足了车辆的起步、加速、爬坡等大转矩低车速的工况。当电动汽车正常行驶时,驱动电机运行于恒功率区,电池组的电流保持大致的恒定。纯电动物流汽车与一般的纯电动汽车相比,具有整车装备质量重、最高车速低、续航里程短、最大爬坡度小等特点,因此纯电动物流汽车的动力系统匹配参数会有所不同。与一般纯电动汽车(15kw-150kw)相比,纯电动物流汽车的功率变化范围大致在30kw-80kw之间,变化范围更趋于平缓。 这就使得纯电动物流汽车运行在低速、制动频繁的城市路况时,电流值过大但是变化范围小、电池负载动态变化平缓以及电池温度偏高等特性,所以对于纯电动物流汽车的电池特性可以对传统的纯电动汽车SOC估计方法进行优化改进,提高纯电动物流汽车SOC估计精度。

如果直接通过实验测量是无法直接得到电池的荷电状态,因此就需要在测量锂电池电压和电流的基础上对电池的SOC进行算法估计\citeup{lizhe2011}。在对锂电池的荷电状态进行估计时需要用到多方面相关的知识,其中包括电化学的基本理论、非线性系统的理论知识以及系统最优参数估计等相关理论。无论是国内还是国外,锂电池SOC估计方法的研究都呈现井喷式的爆炸增长,广大学者们提出了许多优秀实用的估计算法理论,其中比较具有代表性的有开路电压法和安时积分算法,但是传统的SOC估计方法存在需要静止不能在线估计、需要大量时间、存在累计误差等问题。针对以上传统SOC估计方法存在的不足与缺陷,很多学者在SOC估计过程当中引入了基于扩展卡尔曼滤波理论的算法,但是基于扩展卡尔曼滤波的SOC估计算法的雅可比矩阵计算相对复杂,而且在进行线性化的处理中会带来一定的误差。基于无迹卡尔曼滤波的估计算法采用UT变换的方式获取一定数量的采样点逼近非线性系统的密度函数以此来得到状态量的滤波值。UKF不用计算雅可比矩阵,也不会因为线性化而引入额外的误差\citeup{maoqun2010}。UKF虽然解决了扩展卡尔曼滤波算法计算复杂的问题,但是存在对电池模型依耐性强、受初值影响大等问题。针对以上问题,本文拟引入分数阶无迹卡尔曼滤波算法,依靠分数阶微积分的记忆特性来改善UKF的初值问题、模型依赖问题,提高纯电动物流汽车SOC的估计精度。

本论文以四川省科技厅重大产业链项目“纯电动城市物流配送车产业链关键技术研究及运营示范”为依托,针对纯电动物流车电池SOC估计方法进行了深入研究,对电池的静态性能和动态性能进行比较分析,对动力电池模型参数进行辨识,并且对分数阶微积分和无迹卡尔曼滤波的融合进行探索实践。
\section{国内外研究现状}
	\subsection{纯电动汽车的车型概况}
	纯电动汽车的发展始于19世纪80年代,现如今正在全球范围内如火如荼的快速发展着,几乎每天都有关于纯电动汽车的新的技术问世,其中以日本和美国在电动汽车领域的技术最为超前,美国的通用和福特集团、日本的丰田本田等企业都投入了大量的资金在纯电动或者混合动力汽车的研发上。我国在纯电动汽车领域研究的开始相对较晚,政府有关部门也意识到了问题的严重性,所以现在国家大力的推广纯电动汽车\citeup{luoxiaowen2008}。随着政府加大激励力度以及出台相关政策扶持,电动汽车迎来快速发展。早在2012年6月份的时候,国务院就发布了与新能源汽车相关的产业发展总体规划\citeup{xuwenhong2012},在该规划中指出加强新能源汽车关键核心技术研究,努力提升汽车产业水平。目前,国内的一汽、比亚迪、北汽、江淮以及国外Tesla等企业都在从事纯电动汽车研发,已经上市的电动车有Tesla的MODEL S、比亚迪秦和唐、江淮IEV、奇瑞EQ等。现在对于动力电池SOC的估计也以上述车型建模为主,这些车型的功率在100KW以上,最高速度也大于100km/h。本文的研究对象是重汽王牌的纯电动厢式运输车CDW5070XXYH1PEV,发动机的峰值功率为55KW,最高车速为80km/h。对于这种类型车型的研究目前还相对较少,只有对相似的车型纯电动客车动力总成控制策略和整车性能的研究。
	\subsection{动力电池研究现状}
	纯电动汽车是靠车载的电池组作为动力源来驱动的汽车,可以说动力电池组就相当于整部电动汽车的心脏。动力电池的进步可以推动纯电动汽车的发展,降低纯电动汽车的推广成本。
在电动汽车电池的发展历史中主要用到了铅酸电池、镍氢电池和锂离子电池等。其中由于锂电池的高安全性、高能量比并且无污染等特点,以锂材料作为正极材料的电池现在成为了纯电动汽车最广泛采用的电池。在纯电动汽车行业中,主要用到的锂电池有三种:磷酸铁锂电池、锰酸锂电池和三元材料锂电池。磷酸铁锂电池主要由中国和美国的电动汽车生产厂家在主力推广,主要包括比亚迪、中信国安等电池生产厂家,磷酸铁锂电池的主要优点是容量都比较适中,循环寿命高并且高温的放电性能优秀,缺点是低温性能不好并且价格不便宜;锰酸锂电池主要是日本电动汽车生产厂家在推广应用,主要包括日亚化学、日产羚风都企业。锰酸锂的主要特点是低温效果优秀并且价格便宜安全性能也很突出,缺点是循环使用的次数低。三元材料是三种材料中价格最贵的,循环寿命和能力比都十分优秀,但是目前掌握的技术还不够成熟,成本太高。综合上文所述,动力电池必须要满足一定的技术特点才适合用于纯电动汽车。首先动力电池需要有相对比较高的能量密度,这样可以降低电池组的重量以及整车的重量;然后就是电池需要有比较高的温度适用范围,可以满足纯电动汽车在不同的环境下工作行驶;还有就是电池需要安全可靠,不易燃烧和爆炸,这样在极端情况下可以保证行车人员的安全。本文所研究的磷酸铁锂电池基本满足上述的要求条件,在纯电动物流汽车上得到了应用。
	\subsection{等效电路模型研究现状}
	关于锂电池模型的研究主要分为两个大的方向:一个是锂电池的电化学原理模型;另一个是基于经验的模型。电化学模型主要考虑的是电池发生充电和放电反应时,电池内部复杂的化学变化,包括了电池正负极的反应方程、电解质的扩散过程、电池的极化反应特性以及这些反应的衍生现象。电化学模型经过多年的发展也取得了很多显著的成果,锂电池的电化学模型主要分为单粒子模型、Shepherd模型、准二维数学模型以及Unnewhr模型等。其中,单粒子模型的主要思想是将锂电池的正负电极粒子化,将认为电池两端间的扩散反应都只发生在粒子的内部,可以看出单粒子模型表示相对简便,所以粒子模型的使用范围会比较狭窄。

电池等效电路模型是通过电路理论的知识客观全面的描述动力电池能量状态、内部特性和响应特性,正确的电池等效电路模型是纯电动物流车电池SOC估计精确度的基础和保证。在动力电池SOC的估计中经典等效电路模型有以下几种:Rint模型、Thevenin模型、GNL模型和PNGV模型等,后来的研究者们提出的电池等效电路模型基本是对经典模型的扩展和完善。其中比较有代表性的有二阶RC环等效电路模型,该模型将GNL等效电路模型进行了简化,降低了1个模型的阶数,这使得计算电池响应和估算动力电池的内部状态时效率更高,并且可以对电池进行大数据量的参数辨识实验。同时该模型又可以充分的描述电池的动态特性,保持电池的非线性,使得具体工程实现是可行并且有价值的。有研究者提出了高阶的PNGV模型,通过在原有的PNGV模型中增加一个RC环节来描述电池的极化反应,可以更精确地模拟出电压缓慢变化的趋势和电极材料中阻抗的变化。对于确定的电池等效电路模型,一般采取最小二乘法来对模型的参数进行辨识。但是模型的参数是会随着放电倍率、极化电压和电池健康状态而变化的,所以不变的模型参数会降低模型的精度。
	\subsection{基于等效电路模型的SOC估计方法研究现状}
	在纯电动汽车领域的研究中,对电池荷电状态估计的研究一直是待解决的关键问题和难点,电池荷电状态估计的精确度直接影响着电池组的利用率和电动汽车的性能。对于电动汽车SOC估计的方法大体上可以分为两类:一类是根据测量实验推导的非模型方法;另一类是基于电路模型进行测量的估计方法。非模型算法主要包括了安时积分法、直接放电测量法和开路电压测试法。基于模型的算法主要包括维纳滤波的算法、卡尔曼滤波算法以及基于神经网络的算法等。
安时积分法的显著特点是简单易行,该算法的基本思想是通过积分的方式记录下从开始电池开始使用到当前时间所有的耗电量,并且通过与电池起始容量的差值就可以得出当前的剩余电量。但是由于该方法是通过积分的方式,所以误差也会不断的被保留积累下来,所以算法后期运行的精确度会有所下降,导致较大的误差。

开路电压法的基本原理是电池的荷电状态和电池的开路电压之间存在着隐含的函数关系,因为这两者都是直接与电池内部的材料以及电池当前的自身状态相关的。开路电压法首先要完成锂电池SOC-OCV曲线的标定,推导出两者之间的函数方程,然后就可以通过当前时刻的开路电压求出锂电池的当前荷电状态。开路电压法有一个显著的缺点是不能进行在线实时的SOC估计,因为开路电压的测量必须是要等到电池足够静止的状态下测量。

	基于等效电路模型的SOC估计方法中主要运用的是卡尔曼滤波算法和粒子滤波算法。对卡尔曼滤波估计算法的改进主要有两种:基于扩展卡尔曼滤波(excluded kalman filter,简称EKF)理论和基于无迹卡尔曼滤波(unscented kalman filter,简称UKF)理论的SOC估计算法。基于扩展卡尔曼滤波的估计算法是为了将卡尔曼滤波算法应用于非线性的动力电池系统而对卡尔曼滤波算法做的一种改进,其核心思想是将非线性的系统进行线性化处理,通过输入和输出数据拟合出一个与原非线性系统近似的线性化系统,然后再对线性化系统运用卡尔曼滤波算法的思想进行处理。基于扩展卡尔曼滤波的SOC估计算法的雅可比矩阵计算相对复杂,而且在进行线性化的处理中会带来一定的误差\citeup{liuhao2010}。基于无迹卡尔曼滤波的估计算法采用UT变换的方式获取一定数量的采样点逼近非线性系统的密度函数以此来得到状态量的滤波值。UKF不用计算雅可比矩阵,也不会因为线性化而引入额外的误差。粒子滤波器适用于任意的非线性系统,在动力电池的SOC估计中可以取得比较好的估计精度,是一个真正意义上的非线性状态估计器。但是粒子滤波器存在的固有的粒子匮乏特性以及相对较大的计算量使其在SOC估计的实际应用中表现出较差的性能和精度。
	\subsection{基于分数阶微积分SOC估计方法的研究现状}
	分数阶微积分理论与经典的牛顿微积分理论不同,是将经典的微积分理论做了任意阶次的推广,从而扩大了微积分的应用范围,因此近年来关于分数阶微积分的研究如雨后春笋般蓬勃发展。在当前分数阶微积分的应用研究中,分数阶微积分已经在各个领域取得了很好的发展,其中包括物理材料领域、信号处理领域、生物医学研究领域等,并且在这些领域的实际应用中取得了不错的效果。

在信号处理方面,分数阶微积分可以用于分数阶模型系统参数的辨识,根据系统的输入和输出来确定模型的未知参数向量;其次就是分数阶微积分可以构建信号处理中的时延器并且通过分数阶的采样策略来构建新的信号;然后就是分数阶微积分在信号处理方面的应用还可以表现为可以进行分数阶滤波器的设计来降低处理过程的误差。
在物理材料领域,分数阶微积分的理论也起到了举足轻重的作用。例如对于动力学中不同扩散现象的描述,能够使得构架的扩散模型更加的准确,具有更大的包容性。然后还有基于分数阶的布朗运动描述、对流散布建模、Cauchy问题求解、电极描述等问题都得到了深入的研究和发展。

在控制系统领域,许多研究人员从不同的角度对控制系统进行了分数阶的推广,主要包括了分数阶控制系统的描述问题,以及现有的控制理论向分数阶推广的的步骤和方法,还有就是建立了分数阶模型后如何进行模型求解和分数阶模型参数辨识的问题。其中,自适应控制、柔性控制和多物理系统也成为了分数阶理论扩展研究的热点和难点。

总之,从上述可以看出分数阶微积分理论的研究几乎覆盖了所有的领域,但是分数阶微积分在动力电池SOC估计方面的应用研究还相对较少,并且现在的研究主要集中在基于分数阶理论的动力电池动态模型上,将分数阶微积分理论直接应用于SOC估计算法的研究还不够成熟。四川大学的蒲亦非博士等人对于分数阶微积分在现代信号处理中的运用进行了深入的研究,指出了分数阶无迹卡尔曼滤波的可行性,所以将分数阶无迹卡尔曼滤波技术引入并用于动力电池的SOC估计是可行的,具有实际的价值意义。
\section{本文研究目标与研究内容}
本论文围绕纯电动物流车电池SOC估计的精确度进行研究,主要工作包含以下几个方面:
\begin{enumerate}
\item分析纯电动物流汽车在功率、扭矩、电池组容量、电池组电流电压特性与一般的电动轿车的区别,根据电动物流车的这些特性,引入充放电倍率、温度、电池的循环寿命、电池容量等约束因子,建立电动物流车的整车模型。
\item分析比较了目前常用的动力电池等效电路模型,对Rint模型、PNGV模型、GNL模型等模型的优缺点进行总结归纳。并在对二阶RC等效电路模型进行扩展优化的基础上,考虑到电动物流车电池状态的多变性,引入二维动态参数来表征动力电池的特性。
\item运用分数阶微积分的知识对无迹卡尔曼滤波(UKF)进行优化,将分数阶微积分的记忆特性和无迹卡尔曼滤波的反馈特性相结合,更准确地模拟出电池的动态特性,以达到提高电池SOC估计精确度的目的。并在电池动态参数模型上实现基于UKF的SOC估计和基于分数阶微积分无迹卡尔曼滤波的SOC估计,将两种SOC估计效果进行比较验证。
\item引入ADVISOR仿真环境,搭建电动物流车的整车模型并匹配相应参数,选取不同的城市道路实际工况进行整车仿真实验,并基于ADVISOR所提供的数据在Matlab中进行两种卡尔曼滤波算法SOC估计效果的对比验证。
\end{enumerate}

为实施上述内容,规划了如图1-1所示的技术路线图:
\pic[htbp]{研究方案路线图}{}{1-studyroad} 
通过对纯电动物流汽车电池组电路特性进行分析,深入研究磷酸铁锂电池的动态特性,采用锂电池组的二阶RC等效电路模型来表征电池组特征参数,引入分数阶微积分的方法计算等效电路的电流和电压特性,在此基础上对基于卡尔曼滤波的SOC估计算法进行深入研究,采用无迹卡尔曼滤波算法对纯电动物流汽车的电池进行SOC估计,以提高纯电动物流汽车电池SOC估计精度。
\section{论文的结构安排}
本文主要研究纯电动物流汽车电池组的非线性特性、电池组的数学模型与参数辨识、分数阶微积分和无迹卡尔曼滤波估计SOC等内容,主要章节内容安排如下:

第一章:绪论。首先对纯电动物流汽车电池SOC的背景和国内外研究现状进行了总结并与一般的纯电动轿车进行比较分析,进而得出纯电动物流汽车电池组的特点,从而得出本论文的研究内容、技术路线及结构安排。

第二章:磷酸铁锂电池数学模型的研究。基于电化学理论分析锂电池的电化学过程和非线性特性,在此基础上比较了几种等效电路模型,并验证本文所选择的二阶RC模型的有效性和准确性,然后利用最小二乘法对锂电池等效电路模型中的参数进行动态的辨识。

第三章:磷酸铁锂电池SOC估计方法的研究。首先对扩展卡尔曼滤波的理论进行介绍,研究将无迹卡尔曼滤波引入到锂电池SOC估计的方法,分析分数阶微积分理论在锂电池等效电路中的运用,基于分数阶的无迹卡尔曼滤波对锂电池进行SOC估计并与一般的无迹卡尔曼滤波SOC估计进行对比,以此来表明分数阶无迹卡尔曼滤波的SOC估计精度。

第四章:基于ADVISOR的SOC估计算法仿真实验。在ADVISOR平台下搭建中国重汽集团成都王牌商用车公司旗下的纯电动厢式运输车(CDW5070XXYH1PEV)的整车模型,将模型运用于不同的实际道路工况下来进行仿真实验,并根据ADVISOR仿真平台得到的仿真数据来进行锂电池SOC估计的仿真实验。

第五章:总结与展望。总结本文在纯电动物流汽车SOC估计领域所做的工作,描述达到的目标以及存在的不足,并对未来在此领域的研究路线和方向进行展望。

本论文的结构安排如图\ref{1-zuzhi}所示。
\pic[htbp]{论文组织结构图}{}{1-zuzhi} 
	

	
	
