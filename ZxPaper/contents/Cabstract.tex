% !Mode:: "TeX:UTF-8"

\begin{Cabstract}{纯电动物流汽车}{SOC}{电池荷电状态}{分数阶模型}{无迹卡尔曼滤波}
本论文以四川省科技厅重大产业链项目“纯电动城市物流配送车产业链关键技术研究及运营示范”为支撑。由于电动物流汽车采用的磷酸铁锂电池的化学反应非常复杂并且具有高度的非线性性和衍生变化,使得锂电池的荷电状态估计存在一定的不可控性和误差性,因此锂电池荷电状态估计有其自身的难度和复杂性。值得注意的是纯电动物流汽车的安全性、锂电池循环使用的经济性以及续航里程信息等方面又对锂电池荷电状态估计的准确性提出了迫切的要求。所以锂电池的荷电状态估计成为纯电动汽车研究的关键之所在,提高磷酸铁锂电池荷电状态估计的精度值得进一步的深入研究。

首先针对纯电动物流汽车锂电池的荷电状态,以提高估计算法的精度为目的进行研究,构建了锂电池模型和整车模型进行荷电状态估计算法的设计。然后深入研究了磷酸铁锂电池的电化学反应特性,考虑了伴随锂电池反应的电化学极化和浓差极化现象,通过磷酸铁锂电池的容量、内阻、开路电压和温度等特性引出对电池等效电路模型的表征,并且根据实际的计算能力和精度指标要求,综合确定等效电路模型的复杂度以及支撑模型阶次的选取。其次根据锂电池电化学反应的分数阶特性,将等效电路模型推广到分数阶,并且完成分数阶微分等效电路模型的离散滤波器近似与求解,然后依据递推最小二乘法的辨识思想实现对分数阶微分等效电路模型参数的辨识估计。再者对卡尔曼滤波算法用于荷电状态估计的理论进行了研究。在锂电池分数阶等效电路模型的基础上,将扩展卡尔曼滤波和无迹卡尔曼滤波对于非线性系统处理的不同特点对比分析,根据UT变换的思想提出了一种基于分数阶微分模型的无迹卡尔曼滤波算法,从而避免了扩展卡尔曼滤波由于泰勒级数展开忽略高阶项而引入的误差。本算法在一般荷电状态估计算法的基础融合分数阶微积分的记忆特性来达到提高估计算法精度的目的,并且对改进前后的两种荷电状态估计算法进行了对比实验仿真分析,以实验结果来表征估计算法的优越性和精确度。

	最后在ADVIDOR车辆仿真平台下搭建了纯电动物流汽车的整车模型,该模型以成都王牌汽车有限公司生产的纯电动厢式运输车(CDW5070XXYH1PEV)为原型,并且在典型的道路工况下对车辆运行过程进行仿真分析,实验证明了本文提出的磷酸铁锂电池荷电状态估计算法的精确性和有效性。
\end{Cabstract}
